NGC 2903:

Distance taken is 6.4 (this is fine)
at 6.4, scale should be 2.1 pm .5 so we are good. (close enough)

our data:
nstarsin = 8.12463*10^10;
Rin = 2.81221;

Lumin: at 6.4 is 3.66*10^10, so 
M/L=2.2  (perfect)

we could shrink scale length but this galaxy will work.

New Fit:

nstarsin = 7.512463*10^10;
Rin = 2.51221;

This hits the inner region better as desired and lowered the scale length as I wanted.  This will be the best fir for it i think.

M/L=2.05

Synth:

nstarsin = 7.4247355032282788`*^10;
Rin = 2.5304877261188177`;

M/L=2.02

-----

NGC 3521:

Walter m(HI)=80.2*10^8 at D=10.7

`
James Data scaled to D=12 150;R_0=3.29299, lum=4.76853*10^(10)
hence M/L=2.13

This galaxy is perfect in every way.

DONE.

--------

NGC 2841:

our data:
nstarsin = 2.94803*10^11;
Rin = 4.84192;

Walter m(HI)=85.8*10^8 at D=14.1

Leroy R_0= 4.0 at D=14.1

L=4.74242*10^(10)

scale is currently at 4.8 so a little high, but not terrible.  there is so much uncertainty in this galaxy.

gives M/l=6.25 this is fine.

New Fit:

nstarsin = 2.494803*10^11;
Rin = 4.084192;

M/L=5.28

I am impartial,we can use either, but this one is more accurate on scale.


Redo:

nstarsin = 2.594803*10^11;
Rin = 3.766084192;

M/L=5.49

---------

NGC 7331:

James Data D=14 214;R_0=3.1909
Lum== 6.77326*10^(10)

Scale is ok due to begeman, we are in range.


so M/l=1.34 perfect.


Synth:

nstarsin = 1.0645231761597615`*^11;
Rin = 2.8281594289425604`;

M/L=1.57

great.

redo:

9.998777*10^10;
Rin = 2.982281594289425604`;

M/L=1.49


---------

**Ngc 3198 ** not in excel file currently and this is a new run with output below.


D=14.1
nstarsin = 5.86939426221915786716`*^10;
Rin = 4.1004283863;

Lum 3.24101*10^(10)

M/L =1.8 

Perfect, we need to keep this one.

Synth:

nstarsin = 6.09915786716`*^10;
Rin = 4.41004283863;

M/L=1.38

-------
NGC 2403  ** also not in excel file:

nstarsin = 1.20383*10^10;
Rin = 2.17825;

References:
Leroy R_0= 1.6 at D=3.2

de Blok R0= 1.81 at D=3.2

Begeman 2.05 kpc at D=3.25 mpc from Wevers

D=3.25 (checked)

lum=9.25*10^9

gives M/l=1.12  Perfect.

We need to include this one as well.

------

UGC 0128

D=64600;L=5.97259*10^9;R_0=6.89067 James Data

DeBlokMcGaugh1977 give scale =6.8 at 60 so at 65, max scale is around 8.  Currently we are at 10, so we may need to refit this.

Current M/L = 1.58

This one will work, there is enough wiggle room.  we can keep it for now as is.


New fit:

Our data:
nstarsin = 6.447624801648668`*^10;
Rin = 8.13746;

M/L=1.32

PERFECT!!!

Updated LUM=4.6E10Msol
so M/L=2.62 for Xue/Sofue

------

UGC 7524 or NGC 4395 (same galaxy, 2 names)

D=4118;L=3.74314*10^9;R_0=2.70611  James Data
Swaters 2009 gives scale length=2.58 at 3.5=D


nstarsin = 7.89504*10^9;
Rin = 3.32036;

These are fine.

M/l= 2.10

---------

NGC 3992

Tully et al AJ 112, 2471 (1996) gives mean scale length (45.7+43.3+39.9+45.9)/4.= 43.7
arcsec at D=15.5, i.e = 0.290888*15.5*43.7/60.= 3.28388 kpc is the average at D=15.5

This data is at 20.5 so average scale length is 4.5

Our fit:
nstarsin = 1.84417*10^11;
Rin = 4.77053;

So this is fine.

Lum=7.5 at this distance, so

M/L=2.45

This one is fine as is.

Re-Run:

nstarsin = 1.9664417*10^11;
Rin = 4.9597053;


M/L=1.81

Rev2:

nstarsin = 1.79664417*10^11;
Rin = 4.5597053;

M/L=2.39

We are not going to hit these inner points unless they shift left more.  So we may need to stick with one we already have.  Let me know.

Rev3:

nstarsin = 1.75852536`*^11;
Rin = 4.66054;

M/L=2.33
-------

NGC 4138

Our fit:
nstarsin = 3.07227*10^10;
Rin = 1.46121;

actual verheijen data:
D=15 567;L=8.27104 � 109;R_0=1.1849

This is acceptable.

Our M/L:

M/L=3.67

This is fine.


re-Run:

nstarsin = 3.43507227*10^10;
Rin = 1.5466121;
L=1.178

M/L=2.91



------

NGC 3953

Verheijen gives R0(B)= 0.72'=43.2" 

Verheijen gives R0(K)= 0.71'=42.6"

In[27]:= 0.290888*18.7*0.71

Out[27]= 3.86212 reported scale at 18.7

But distance is taken at 15.0, hence scale: 3.0

Our fit:

nstarsin = 5.24365*10^10;
Rin = 2.60155;

lum from verheijen: 2.91*10^(10)


This is within bounds, but we can tighten it up if we like.

M/L=1.78694

Good.

New Fit:

nstarsin = 8.240778337871786`*^10;
Rin = 3.6452155;

L=4.19


M/L=1.966

Great.

Rev2:

nstarsin = 7.852536`*^10;
Rin = 3.36054;

m/l=1.87

-------

UGC 6973

D=25 260;L=1.64663 � 1010;R_0=2.80444 James Data
actual data taken at 15.0
scale range from Tully .78-2.0



our data:
nstarsin = 8.93866*10^9;

Rin = 0.754914;

These are good.

Lum=0.62*10^(10) at 15.0

M/L=1.44

Great.

New Fit:

nstarsin = 1.0237269600366442`*^10;
Rin = 0.85684;

L=.8928

M/L=1.15

------

NGC 4088

our data:

nstarsin = 3.388907*10^10;
Rin = 2.70255;

data taken at 15.5, scale range 2.4-2.8 verheijen

Lum 0.61*10^(10) at 15.5

M/L=5.58

Great, as is.


New Run:

nstarsin = 5.691520002157881`*^10;
Rin = 3.646;
L=.8784

M/L=6.48

------

NGC 3726:

our data:

nstarsin = 2.804638*10^10;
Rin = 2.99291;

scale:
Verheijen gives R0(B)= 0.97'=58.2" 

Verheijen gives R0(K)= 0.63'=37.8" range is around 2.8-3.2

lum  2.65*10^(10) at 15.5

M/L=1.06

perfect

Re-Run:

nstarsin = 3.1804638*10^10;
Rin = 3.2688;

Lum= 3.77*10^10


M/L=.85

Great.

Synth:

nstarsin = 3.271359543046786`*^10;
Rin = 3.28901;

M/L=0.86

Synthrev:

nstarsin = 4.0252276807530918`*^10;
Rin = 3.2128901;

M/L=1.067

Remix: 

nstarsin = 4.75252276807530918`*^10;
Rin = 4.02128901;

M/L=1.26

Remix 2:

nstarsin = 3.9175252276807530918`*^10;
Rin = 4.02128901;
M/L=1.03

-----
NGC 5055

distance is 10.1
our data:

nstarsin = 1.05602*10^11;
Rin = 3.28688;

Leroy R_0= 3.2 at D=10.1

so scale is SPOT on!

Lum=4.3479859*10^(10)

M/L=5.87

Perfect


Rerun:

nstarsin = 1.061785602*10^11;
Rin = 3.00993;

Lum=3.62187*10^(10)

M/L=3.06


New Re-run:

I found a new reference to get you the scale you wanted:  Battaglia has 3.4 at D=7.2.  Thus, we can go as high as 4.3 (H ? study of the warped spiral galaxy NGC 5055: a disk/dark matter halo offset?
Giuseppina Battaglia1, Filippo Fraternali2, Tom Oosterloo3, and Renzo Sancisi4)

nstarsin = 1.1061778*10^11;
Rin = 4.309928688;

M/L= 3.05

Synth:
nstarsin = 1.0875113377740219`*^11;
Rin = 3.3032034386469724`;

M/L=2.48344



-------

NGC 5533

data input at d=54.

our data:

nstarsin = 2.2294343*10^11;
Rin = 3.231;

But use Kent1985  (7.10 kpc at 39.4, i.e. 7.10/39.4*42.= 7.56853) or
 36.2 arc sec to get 7.4 kpc scale length at D=42. 
 
 Thus our scale is way off on this one.  I did another run which misses the bulge, to get:
 

nstarsin = 3.2294343*10^11;
Rin = 7.231;

lum=4.5*10^10

M/L=7.11

This is fine with our new fit.

Redo:
nstarsin = 2.412294343*10^11;
Rin = 7.4231;
M/L=5.36

Redo 2:

nstarsin = 3.412294343*10^11;
Rin = 7.04231;
M/L=7.52

-----

NGC 5907
data is at 15.5

our data:
nstarsin = 1.02997*10^11;
Rin = 3.45303;

Barnaby 1992 K band R_0=4.0 at D=11, i.e R_0=6.0 at D=16.5 
and H band R_0=3.70 at D=11, i.e R_0=5.55 at D=16.5

overall we are good on scale.

lum=5.0 at 15.5

M/L=2.04

Great.



Re-run:

nstarsin = 1.3232997*10^11;
Rin = 5.045303;

Lum=7.2

M/L=1.84



--------


NGC 6946

data taken at 10.1
our data:

nstarsin = 6.848955*10^10;
Rin = 5.12416;

scale should be 4.3 at 10.1 according to Leeroy:  Leroy R_0= 2.5 at D=5.9

this is not so far out, but we should maybe scale it a little better.

Lum=7.99722*10^10

M/L=.85

NEW RUN (rev)

Our data:
nstarsin = 4.39848955*10^10;
Rin = 2.5080416;

Lum=2.72898*10^10 at this distance.

M/L=1.929

Perfect.  This one is SOOO good now.

Second rerun:

nstarsin = 5.339848955*10^10;
Rin = 2.580416;

M/L=1.96

Synth:

nstarsin = 5.497531336837891`*^10;
Rin = 3.0669816428616086`;

Great.

M/L=2.01


Synth: McGaugh

nstarsin = 3.868723903825065`*^10;
Rin = 3.0439380664594347`;

M/L=1.41

-------

NGC  6946 Mcgaugh run

Our data
nstarsin = 3.85813*10^10;
Rin = 3.1113;

Perfect on the scale at 6.9.

M/L=1.41

Synth:

nstarsin = 3.868723903825065`*^10;
Rin = 3.0439380664594347`;

M/L=1.41
-------------
NGC 925
distance at 9.25
scale Leroy R_0= 4.1 at D=9.2
James D=8700;R_0=3.87717

our data:
nstarsin = 2.47912*10^10;
Rin = 6.3473;

lumin: 1.61436*^10

NEW RUN to fix scale length:
nstarsin = 1.47912*10^10;
Rin = 4.3473;

this one is accurate.

M/L = .919

we should use the new run if possible.

Synth:

Identical to last:

nstarsin = 1.47912*10^10;
Rin = 4.3473;

M/L=.919

Synth:

nstarsin = 1.747912*10^10;
Rin = 4.3473;

M/L=1.085

Synth Rerun;

nstarsin = 1.3747912*10^10;
Rin = 4.3473;

M/L=0.85

--------------

F-563-1
distance at 45.

our data:

nstarsin = 1.426221915786716`*^10;
Rin = 2.83863;

DeblokMcGaugh1997 gives scale=2.8 at 45.0

SCale is perfect.

lum: DeblokMcGaugh1997 gives Abs Mag(B)=-17.3 at 45.0

In[1]:= 10^(5.48/2.5 + 17.3/2.5)

Out[1]= 1.2942*10^9

Hence, M/L=1.13

Perfect.

Synthrev 2:

nstarsin = 1.796426221915786716`*^10;
Rin = 2.6;

M/L=1.39

------------

M-33

data in at d=.84

our data:
nstarsin = 5.0472943*10^9;
Rin = 1.45962;

scale info In text Kent AJ 94, 306 (1987)  gives scale = 9.6'=576" a4 d=.84
In[14]:= 0.290888/60.*576.*0.84

Out[14]= 2.34572

With 2.5Log[10,e]=2.5*.434294=1.086, to digitize set R_0=1.086
(R_2-R_1)/(\mu_2-\mu_1).
from Figure and Table 3 get  R_0=1.086(13.18-2.92)/(22.13-20.80)=8.38 arc
min.

this corresponds to a 2.0 min scale length.

Hence it seems like we will need to make a re-run of this.

New Run at min. scale length:
nstarsin = 6.60000472943*10^9;
Rin = 2.0;

lum=0.1*10^(10)

Hence for new run,

M/L=6.6


OK, ignore the New scale, since we are in range with the new reference.  Hence we keep original run:

our data:
nstarsin = 5.0472943*10^9;
Rin = 1.45962;

M/L=5.047
(this one best for XueSofue..too bad.)
Synth:

nstarsin = 5.549102695187223`*^9;
Rin = 1.415962;

M/L=5.54
using this one--is better than bigger scale below

Redo 2:

nstarsin = 5.2897549102695187223`*^9;
Rin = 1.766415962;
M/L=5.28

-----

NGC 7793

our data:
nstarsin = 7.9010*10^9;
Rin = 1.51432;

Dicaire:
R=1.1+2
L=3.1*10^9
D=3.38

New Run to hone in the scale:

nstarsin = 6.9610*10^9;
Rin = 1.31432;

Great.  as good as we are going to get with this shape and it checks with the dicaire paper.

M/L=2.245

Synth:

nstarsin = 6.73424707687294`*^9;
Rin = 1.363751432;

M/L=2.243

Synthrev 2:

nstarsin = 8.1294`*^9;
Rin = 1.1521111;
M/L=2.61

-------

NGC 891

our data:

nstarsin = 6.62743*10^10;
Rin = 2.35299;

Frat: R=4.18-5.13
D=9.5
L=2.5*10^10

Freeman says we can go down as low as 3.5.

New fit for scale bigger:

nstarsin = 7.62743*10^10;
Rin = 4.135299;

Note, this galaxy is also almost completely edge 88 degree inclination,  on (the worst kind of remeasuring rotation curves).

NOTE: this fit looks terrible, but this galaxy is actually ok.  if you read the paper, Frat states that this galaxy is DOMINASTED by bulge, and the disks do almost nothing to hit the profile.  But if you match ours to that in the paper, we are spot on in this fit.  We can discuss this further in person, but I think we live with it and make a note if it when needed.

M/L=3.048

------

NGC 7814

our fit:

nstarsin = 3.87643*10^10;
Rin = 0.82529;

Frat:
R=4.26
D=14.6
L=1.3*10^10

New Reference:

can be as low as 3.5

Note again this galaxy is almost edge on as well.  They all use a van der kruit and searle law, which we can as well.

Synth:

nstarsin = 5.687643*10^10;
Rin = 3.610582529;

Redo 2:

nstarsin = 5.687643*10^10;
Rin = 1.3710582529;
M/L=3.61

----------

M-31
D=.74

our original fit had a scale length that seemed a bit high.   Here we can get a fit which is right in line with the data.

New Fit:
our data: 
nstarsin = 1.533784*10^11;
Rin = 4.80296;
lum=2.6?1010

M/L=5.88

This one is great.

------

------

Milky Way:  All of these are physical.  The scale of the Mway is between 2.0-2.5 from Mcgaugh

AnatolB:
nstarsin = 5.465653532648248`*^10;
Rin = 2.178946;

Check:
nstarsin = 6.00465653532648248`*^10;
Rin = 4.490178946;

---

AnatolA:
nstarsin = 4.221146016988169`*^10;
Rin = 2.3841655911121227`;

check:

nstarsin = 4.8221146016988169`*^10;
Rin = 3.93841655911121227`;
---

Xue:

nstarsin = 5.3453189582334694`*^10;
Rin = 2.430325553970189`;

Check:

nstarsin = 6.50003453189582334694`*^10;
Rin = 5.330000053430325553970189`;
---

Xue Disk:

nstarsin = 5.436875941675944`*^10;
Rin = 2.5;

Check:

nstarsin = 8.2636875941675944`*^10;
Rin = 4.16;
---

Sofue:

nstarsin = 5.436875941675944`*^10;
Rin = 2.0;

Check:
nstarsin = 8.0436875941675944`*^10;
Rin = 4.430;
